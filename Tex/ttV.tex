\section{ttV}
\label{sec:ttV}


\subsection{Samples}
Two MC generators compared in this study.
The nominal sample for \ttW production was generated using the \textsc{Sherpa}~2.2.1~\cite{sherpa} generator with the NNPDF3.0 NLO PDF set.
The matrix element (ME) was calculated for up to one additional parton at NLO and up to two partons at LO using
\textsc{Comix}~\cite{Gleisberg:2008fv} and \textsc{OpenLoops}~\cite{Cascioli:2011va}, and merged with the \textsc{Sherpa} parton shower~\cite{Schumann:2007mg} using the \textsc{MePs@Nlo} prescription~\cite{Hoeche:2012yf}.
The choice of renormalisation and factorisation scales is $\mu_R = \mu_F = H_\textrm{T}$/2, where $H_\textrm{T}$ is defined as the scalar sum of the transverse masses $\sqrt{p_\textrm{T}^2+m^2}$ of all final state particles.



Systematic uncertainties due to missing higher-order QCD corrections are estimated by varying the factorisation and renormalisation scales in the nominal sample simultaneously by a factor of 0.5 and 2.0 with respect to the central value. 
Uncertainties associated with the modelling of additional QCD radiation are estimated by comparing the nominal \ttW prediction with that of an alternative sample that was generated at NLO (thus at lower order than for the nominal sample) with the \textsc{MadGraph5\_aMC@NLO}~2.2.1 generator using the same scale choice and PDF set as for the nominal sample, and interfaced to \textsc{Pythia}~8.2 in combination with the A14 tune. 
The samples configurations are summarized in Table~\ref{tab:mcconfig}.
%%%%%%%%%%%%%%%%%%%%%%%%%%%%%%%%%%%%%%%%
\begin{table}
\begin{center}
\caption{\label{tab:mcconfig}
The configurations used for the event generation of \ttW processes.}
\vspace{0.25cm}
{\small
\setlength\tabcolsep{1.5pt}
\begin{tabular}{llllll}
\hline\hline
Process & Generator & ME order & Parton shower & PDF & Tune  \\
%& (alternative) & (alternative) & & \\
\hline
$\ttbar W$  & \textsc{Sherpa 2.2.1} & \textsc{MePs@Nlo} & \textsc{Sherpa} &  NNPDF3.0 NNLO & \textsc{Sherpa} default \\
& \textsc{MG5\_aMC} & NLO & \textsc{Pythia} 8 & NNPDF3.0 NLO & A14   \\
\hline\hline
\end{tabular}
}
\end{center}
\end{table}


% Finally, the uncertainty due to the choice of PDF set is evaluated using the PDF4LHC15 prescription.

\subsection{Fiducial Volume}
Object and event selection is defined at particle-level that closely match the detector-level described in reference~\cite{HIGG-2017-02}. 
Jets are reconstructed from stable particles with a mean lifetime of $\tau > 3\times 10^{-11}$~s, using the anti-$k_t$ algorithm with a radius parameter of $R=0.4$.
Jets are required to satisfy $\pt > 25~\gev$ and $|\eta| < 2.5$.
Jets are matched to $b$-hadrons with $\mathrm{p_{T}>25 GeV}$ by ghost matching~\cite{70ttbb} and are referred to as $b$-jets. 
Electrons and muons, referred to as light leptons, are required to be separate from selected jets by $\Delta R>0.4$. 
Events are selected with exactly two light leptons.
Leptons are required to have $|\eta|< 2.5$ and $\mathrm{p_{T}>25(20) GeV}$ for leading (subleading) lepton. 
Leptons are required to have same charge, targeting the semi-leptonic $\mathrm{t\bar{t}}$ decay and leptonic $W$ decay.

%Events separated by Hadronic tau decays Hadronically decaying $\tau$ lepton
 
Five regions are regions are defined in total, categorized by the number of jets (three or  $\geq$4), $b$-jets (one or $\geq$2) as well as the existence of hadronically decaying $\tau$ lepton. 
%The event selection requirements are summarized in Table~\ref{}.
%defined by 5 jets \& 3 b-jets or >=6jets \& >=4 b-jets.

%\subsection{Results}




